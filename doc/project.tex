\documentclass{article}
\usepackage[hmargin=1in,height=9in]{geometry}
\usepackage{algorithm,algpseudocode}
\usepackage{beton,euler}
\usepackage{tikz}
\usepackage{amsmath,amssymb}
\usetikzlibrary{decorations.pathreplacing,shapes.multipart,matrix}
\begin{document}

\title{A REST Inspired Virtual File System}
\author{Louis J. Scoras \texttt{<ljsc@gwu.edu>}\\
Catalino Cuadrado \texttt{<catalino@gwu.edu>}\\
The George Washington University\\
\\
CSCI 6221 - Term Project\\
Professor Bellaachia}
\date{\today}
\maketitle

\begin{abstract}

Over the last several years, Roy Fielding's "Representational State Transfer"
(REST) architecture has had a profound impact on the design of the web; many of
the most heavily frequented webservice APIs have been implemented using the
ideas proposed by Fielding in his dissertation. Although his paper specifically
discusses REST as an implementation pattern for the web---in particular the
paper explores the intricacies of the HTTP protocol---we believe that REST can
be partially applied in other areas as well.

For our project, we will be implementing a virtual file system. We will use its
construction as a vehicle for exploring the efficacy of REST principles in
designing a generic file system. The heart of a REST service is a mapping of a
URI space to logical resources, which the client is interested in accessing.
The primary goal of a virtual file system is likewise the same. Therefore we are
confident that many of these principles can be reused. To wit, it is our
intention to leverage a framework for creating RESTful web applications as the
basis for our implementation.

As writing a file system driver from scratch would be an exhausting
process---one which might distract from the essence of our problem---we will be
making use of the FUSE project. FUSE (File System in User Space), is an API for
writing file systems which don't require kernel level access. It also provides a
greatly simplified interface for simple file systems. In addition, it has
bindings written for several high-level languages, including Ruby. Since Ruby
also hosts many successful MVC web frameworks, this makes for a natural choice.

In serving our goal of demonstrating REST as a good system for creating a file
system, we will use a readily available public web service as the provided
content of the file system. A user will be able to use normal command line and
GUI tools to browse and modify this content. As such, the main body of our
project will be dedicated to mapping HTTP verbs to their appropriate FUSE
counterparts, thereby providing the basic capabilities just described. We will
also attempt to mimic some of the more advanced parts of REST, such as resource
negotiation and caching as well.

\end{abstract}


\end{document}

