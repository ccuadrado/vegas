\documentclass{article}
\usepackage[hmargin=1in,height=9in]{geometry}
\usepackage{algorithm,algpseudocode}
\usepackage{beton,eulervm}
\usepackage{tikz}
\usepackage{amsmath,amssymb}
\usepackage{alltt}
\usetikzlibrary{decorations.pathreplacing,shapes.multipart,matrix}
\bibliographystyle{alpha}
\begin{document}

\renewcommand{\bfdefault}{sbc}

\title{A REST Inspired Virtual File System}
\author{Louis J. Scoras \texttt{<ljsc@gwu.edu>}\\
Catalino Cuadrado \texttt{<catalino@gwu.edu>}\\
The George Washington University\\
\\
CSCI 6221 - Term Project\\
Professor Bellaachia}
\date{\today}
\maketitle

\begin{abstract}

Over the last several years, Roy Fielding's Representational State Transfer
(REST) architecture \nocite{Fie00} has had a profound impact on the design of
the web; many of the most heavily frequented webservice APIs have been
implemented using the ideas proposed by Fielding in his dissertation. Although
his paper specifically discusses REST as an implementation pattern for the
web---in particular the paper explores the intricacies of the HTTP protocol---we
believe that REST can be partially applied in other areas as well.

For our project, we will be implementing a virtual file system. We will use its
construction as a vehicle for exploring the efficacy of REST principles in
designing a generic file system. The heart of a REST service is a mapping of a
URI space to logical resources, which the client is interested in accessing.
The primary goal of a virtual file system is likewise the same. Therefore we are
confident that many of these principles can be reused. To wit, it is our
intention to leverage a framework for creating RESTful web applications as the
basis for our implementation.

As writing a file system driver from scratch would be an exhausting
process---one which might distract from the essence of our problem---we will be
making use of the FUSE project. FUSE (File System in User Space), is an API for
writing file systems which don't require kernel level access. It also provides a
greatly simplified interface for simple file systems. In addition, it has
bindings written for several high-level languages, including Ruby. Since Ruby
also hosts many successful MVC web frameworks, this makes for a natural choice.

In serving our goal of demonstrating REST as a good system for creating a file
system, we will use a readily available public web service as the provided
content of the file system. A user will be able to use normal command line and
GUI tools to browse and modify this content. As such, the main body of our
project will be dedicated to mapping HTTP verbs to their appropriate FUSE
counterparts, thereby providing the basic capabilities just described. We will
also attempt to mimic some of the more advanced parts of REST, such as resource
negotiation and caching as well.

\end{abstract}

\clearpage
\tableofcontents

\section{Problem Statement}

Our project aims to create a virtual file system code-named VegasFS. Vegas will
implement an adapter to the twitter online messaging service. By making use of
REST Principals and the FuSE project Vegas will allow a user should be able to
mount the system to an arbitrary part of the file space they have appropriate
permissions for. Then, using the standard tools for browsing files provided by
the operating system---both command line and GUI---they will be able to access
several different areas of the twitter service.

The specific parts of the api that can be accessed are described in more detail
in the section on project requirements below.


\section{Requirements Analysis}

\subsection{Requirements Overview}

In the following sections we will outline the project requirements. Each
requirement will have a short name, which is indicative of purpose for the
feature and is intended to for identifying the requirement. In addition, there
is a short user story, which explains the functionality in further detail.
Finally, each requirement will have a section which describes the acceptance
test which must pass for the requirement to be accepted.

The section on testing bares further discussion of terminology and notation.
Because the goal of the project is to make a file system for a POSIX
environment, it seems only right to define the desired operation in terms of
standard *NIX command line tool. Although for the project we will be creating an
automated test suite for all specified tests, it should also be possible to
verify correct operation using only tools on the command line.

As such, we will make heavy use of the following tools:

\begin{figure}[H]
\centering
\begin{tabular}{|c|c|}
\hline
COMMAND & DESCRIPTION \\\hline

ls -1 & Output each file in the directory on per line.\\

cat \textit{filename} & Output the contents of \textit{filename} to standard
output.\\

cut -d \textit{sep} -f \textit{field} & Output only the field numbered
\textit{field} using \textit{sep} to separate columns.\\

grep -i \textit{regex} & Output lines from the input that match \textit{regex}.
Case insensitive.\\

wc -l & Read the input and output the number of lines.\\

file \textit{filename} & Determine the file type of \textit{filename} and output\\\hline
\end{tabular}
\end{figure}

We assume familiarity with regular expressions and with UNIX
pipes\footnote{TODO: Add references for refreshers}. In user stories, any path
component prefixed with a ':' is to be considered a wild card, and will be
referenced by the name following the symbol.

\subsection{Static Requirements}

\newcounter{requirements}
\newenvironment{Requirements}
  {\begin{list}{SR-\arabic{requirements}.}%
               {\usecounter{requirements}}}%
  {\end{list}}

\begin{Requirements}

\subsubsection{Getting simple information}
%%%%%%%%%%%%%%%%%%%%%%%%%%%%%%%%%%%%%%%%%%%%%%%%%%%%%%%%%%%%%%%%%%%%%%%%%%%%%%%%
\item Get information for a user.

\textbf{User Story:} A user reads the file to the path \texttt{/:user/info.txt}.
Inside the file they see information regarding the user named \texttt{:user}. It
contains the following attributes:

\begin{itemize}
\item Full name
\item Screen name
\item Followers count
\item Statuses count
\item Location
\item Join date
\end{itemize}

\textbf{How to test:} The following should output "Lou Scoras".

\begin{alltt}
    \$ cat /ljsc/info.txt | grep -i 'Full-Name' | cut -d: -f2 
\end{alltt}

%%%%%%%%%%%%%%%%%%%%%%%%%%%%%%%%%%%%%%%%%%%%%%%%%%%%%%%%%%%%%%%%%%%%%%%%%%%%%%%%
\item Get a tweet given a particular id\label{req:get-a-tweet}

\textbf{User Story:} A user reads the file \texttt{/tweets/:tweet\_id.txt}.
The contents of the file will have the body of the tweet, which will contain:

\begin{itemize}
\item Message
\item Date
\item Author
\end{itemize}

If the user wants only the message contents, they can read the contents of
\texttt{/tweets/:tweet\_id/body.txt} alone.

\textbf{How to test:} Using the following command on predetermined tweet
\textit{n} yields ``This is a canned tweet.``

\begin{alltt}
    \$ cat /tweets/n/body.txt
\end{alltt}

\subsubsection{Getting lists of tweets}
%%%%%%%%%%%%%%%%%%%%%%%%%%%%%%%%%%%%%%%%%%%%%%%%%%%%%%%%%%%%%%%%%%%%%%%%%%%%%%%%
\item Get a user's latest tweets\label{req:get-latest}

\textbf{User Story:} A user lists the contents of the directory
\texttt{/:user/tweets/latest}. For each tweet in the timeline there will be a
file in the directory, \texttt{:tweet\_id.txt}. The contents of this file are as
described in SR-\ref{req:get-a-tweet}.

\textbf{How to test:} Take any tweet in a user's timeline within the last $n$
tweets, all of which are considered the latest. Let this tweets id be
\textit{tweet\_id}. The following command should then output $1$.

\begin{alltt}
    \$ ls -1 /vegasfs/tweets/latest | grep -i "\textit{tweet\_id}.txt" | wc -l
\end{alltt}

%%%%%%%%%%%%%%%%%%%%%%%%%%%%%%%%%%%%%%%%%%%%%%%%%%%%%%%%%%%%%%%%%%%%%%%%%%%%%%%%
\item Get a user's @mentions

\textbf{User Story:} This directory returns a user's mentions, up to 800 tweets (A
limitation by the API). To access the latest mentions, the user can list the
contents of \texttt{/:user/mentions} for the latest mentions for that user.
Also, they can fetch \texttt{/:user/mentions/:j-:k/} to list all the mentions
between the $j$th and $k$th mention in the timeline. Each file in these
directories will be \texttt{:tweet\_id.txt} and have the information as in
SR-\ref{req:get-a-tweet}.

\textbf{How to test:} For each file in the directory of mentions, we can ensure
that they all begin with `@` and contain \texttt{@:username}. The following
three commands should output the same value:

\begin{alltt}
    \$ cat /\textit{:user}/mentions/*.txt | wc -l
    \$ cat /\textit{:user}/mentions/*.txt | grep '^message:@' | wc -l
    \$ cat /\textit{:user}/mentions/*.txt | grep '^message:@' | grep '@\textit{:username}' | wc -l
\end{alltt}

\subsubsection{Updating tweets}
%%%%%%%%%%%%%%%%%%%%%%%%%%%%%%%%%%%%%%%%%%%%%%%%%%%%%%%%%%%%%%%%%%%%%%%%%%%%%%%%
\item Create a new tweet\label{req:create-tweet}

\textbf{User Story:} This feature allows creation of a new tweet by writing to a
file at location \texttt{/tweets/new.txt} with the contents of your tweet. The
date and author information will be appended automatically by the twitter
system.

\textbf{How to test:} Create a new tweet.

\begin{alltt}
    \$ vi /tweets/new.txt  # edit and save
\end{alltt}

The following command should contain the message entered in the previous step.

\begin{alltt}
    \$ cat \texttt{/:user/tweets/latest/}\$(ls -lt1 \texttt{/:user/tweets/latest} | head -n 1)
\end{alltt}


\subsubsection{Searching}
%%%%%%%%%%%%%%%%%%%%%%%%%%%%%%%%%%%%%%%%%%%%%%%%%%%%%%%%%%%%%%%%%%%%%%%%%%%%%%%%
\item Search for a hashtag

\textbf{User Story:} A user searches for a keyword among all tweets via the
hashtag metadata convention. When a user creates a tweet, they can prefix
keywords with the `\#` character. This search will target these tweets. The user
lists the contents of the directory \texttt{/:hashtag/tweets/}. There is one
tweet per file as described in SR-\ref{req:get-latest}.

\textbf{How to test:} Tweet a new tweet, or reuse a previous tweet for this test
as per SR-\ref{req:create-tweet}. In this tweet, create a hashtag with a GUID.
Then search for that tag as in the following command; it should output $1$.

\begin{alltt}
    \$ ls -1 \texttt{/\textit{GUID}/tweets/} | wc -l
\end{alltt}

\subsubsection{Content Negotiation}
%%%%%%%%%%%%%%%%%%%%%%%%%%%%%%%%%%%%%%%%%%%%%%%%%%%%%%%%%%%%%%%%%%%%%%%%%%%%%%%%

\item View tweets with images as images\label{req:images}

\textbf{User Story:} Users should be able to get attached media content from
tweets by changing the file extension. The if the user inspects a file
\texttt{tweet\_id.txt}, they should be able to see the picture embedded in the
tweet by opening \texttt{tweet\_id.png} for instance.

File formats that would be supported would be .png and .jpg. All embedded urls
using the yfrog service will be detected.

\textbf{How to test:} There are command-line utilities to inspect meta data
about an image file. One of these could be used to ensure that the data returned
is a valid image format. Also, we can use the UNIX \texttt{file} command to do
something similar:

\begin{alltt}
    \$ file /:user/tweets/1234.png # should return png
\end{alltt}

%%%%%%%%%%%%%%%%%%%%%%%%%%%%%%%%%%%%%%%%%%%%%%%%%%%%%%%%%%%%%%%%%%%%%%%%%%%%%%%%
\item Open a link for a given tweet as the expanded web page

\textbf{User Story:} If the embedded url is not detected as an image as in
SR-\ref{req:images}, it will be considered a normal url. If the user opens up
\texttt{tweet\_id.html}, it will return the contents of fetching that url
directly.

\textbf{How to test:} Open the file \texttt{tweet\_id.html} as described in the
story. The browser should show you the indicated page.

\end{Requirements}

\subsection{Optional Requirements}

\setcounter{requirements}{1}
\renewenvironment{Requirements}
  {\begin{list}{OR-\arabic{requirements}.}%
               {\usecounter{requirements}}}%
  {\end{list}}

\begin{Requirements}

\subsubsection{Advanced Search}
%%%%%%%%%%%%%%%%%%%%%%%%%%%%%%%%%%%%%%%%%%%%%%%%%%%%%%%%%%%%%%%%%%%%%%%%%%%%%%%%

\item Search for tweets in a geographical area

\textbf{User Story:} A user searches for nearby tweets by giving geotag
information.

The directory \texttt{/search/geo/:lat\_:long} will contain a number of files
with tweet information as in SR-\ref{req:get-latest}.

\textbf{How to test:} To test, the user would list the contents of the file
\texttt{/search/geo/xxx\_yyy/} and inspect the first file. This requirement will
be manually verified by inspecting the meta data for that tweet.

\subsubsection{Caching}

%%%%%%%%%%%%%%%%%%%%%%%%%%%%%%%%%%%%%%%%%%%%%%%%%%%%%%%%%%%%%%%%%%%%%%%%%%%%%%%%
\item Requests for tweets are cached between calls

\textbf{User Story:} A user gets information as in any of the static
requirements as above. If possible, the system should not re-fetch the entire
contents of the data if the cached version is fresh.

\textbf{How to test:} To be determined.

\end{Requirements}

\section{design document}
\subsection{Purpose}
The problem we are trying to solve is how to map a RESTful API onto a POSIX filespace. This allows users to browse the data in a familiar topography and also utilize various command-line tools for browsing, manipulating, creating, and deleting data. 

Our program is meant to be used by an average user who has a basic understanding of file systems and command-line tools. Our goal is to abstract the process so that the user does not need to know the specific RESTful API, he or she simply needs to know simple file operations in order to interact with the program.

\subsection{High Level Entities}
Our design consists of a RESTful API that interfaces with our ruby program. For demonstration purposes we are basing our implementation off of the Twitter API(\textit{http://http://dev.twitter.com/doc})The Ruby middleware uses Sinatra for pairing methods with RESTful routes. We use the ruby implementation of the FUSE (Fileystem in User SpacE) (\textit{https://rubyforge.org/projects/fusefs/}) to actually mimic a POSIX filesystem on the user's local environment. All data will be fetched from the API. We may implement a caching algorithm if time allows, but the program should be designed to allow caching to be implemented.   

\section{Tests}

Tests for the VegasFS projects are provided as executable specifications written in
RSpec, a behavior-driven development tool for Ruby. These specs were used in
the development of the project. They can be used to verify the correct
implementation of the Vegas driver and several of the features exposed via the
Sinatra layer.

To run the specs, simply execute the default Rake\footnote{Ruby's equivalent to
make} task at the command line. You can do this by typing \texttt{rake} from the
command line at the project root. The source code for these specifications can
be found in the \texttt{spec} directory.

As an example of what the specs look like, we have included an excerpt the
specification for the driver below. Also, recall that in
section~\ref{sec:requirements} we have also included how to manually test the
system from the command line.

\begin{verbatim}
describe VegasFS::Driver do
  describe "connected to localhost port 777" do
    before do
      @vegas = VegasFS::Driver.connect(:host => 'localhost', :port => 777)
    end

    describe "accessing a file" do
      it "should submit a GET request on read_file calls" do
        stub_request(:get, 'localhost:777/foo/bar')
        @vegas.read_file("/foo/bar")

        a_request(:get, "http://localhost:777/foo/bar").should have_been_made
      end

      it "should return the body contents of a GET request" do
        stub_request(:get, 'localhost:777/foo/bar').to_return(:body => "Baz!")
        @vegas.read_file("/foo/bar").should == "Baz!"
      end
    end
  end
end
\end{verbatim}

In the example we first set up an instance of the Vegas router. Then we specify
that whenever fuse calls Vegas to read a file, that a GET request should be sent
out to the corresponding URI path on the Sinatra handler.



\addcontentsline{toc}{section}{References}
\bibliography{project}

\end{document}

