\section{Requirements Analysis}

\subsection{Static Requirements}

In the following sections we will outline the project requirements. Each
requirement will have a short name, which is indicative of purpose for the
feature and is intended to for identifying the requirement. In addition, there
is a short user story, which explains the functionality in further detail.
Finally, each requirement will have a section which describes the acceptance
test which must pass for the requirement to be accepted.

The section on testing bares further discussion of terminology and notation.
Because the goal of the project is to make a file system for a POSIX
environment, it seems only right to define the desired operation in terms of
standard *NIX command line tool. Although for the project we will be creating an
automated test suite for all specified tests, it should also be possible to
verify correct operation using only tools on the command line.

As such, we will make heavy use of the following tools:

\begin{figure}[H]
\centering
\begin{tabular}{|c|c|}
\hline
COMMAND & DESCRIPTION \\\hline

ls -1 & Output each file in the directory on per line.\\

cat \textit{filename} & Output the contents of \textit{filename} to standard
output.\\

cut -d \textit{sep} -f \textit{field} & Output only the field numbered
\textit{field} using \textit{sep} to separate columns.\\

grep -i \textit{regex} & Output lines from the input that match \textit{regex}.
Case insensitive.\\

wc -l & Read the input and output the number of lines.\\

file \textit{filename} & Determine the filetype of \textit{filename} and output\\\hline
\end{tabular}
\end{figure}

We assume familiarity with regular expressions and with UNIX
pipes\footnote{TODO: Add references for refreshers}. In user stories, any path
component prefixed with a ':' is to be considered a wildcard, and will be
referenced by the name following the symbol.

\hrulefill

\newcounter{requirements}
\newenvironment{Requirements}
  {\begin{list}{SR-\arabic{requirements}.}%
               {\usecounter{requirements}}}%
  {\end{list}}

\begin{Requirements}

\subsubsection{Getting simple information}
%%%%%%%%%%%%%%%%%%%%%%%%%%%%%%%%%%%%%%%%%%%%%%%%%%%%%%%%%%%%%%%%%%%%%%%%%%%%%%%%
\item Get information for a user.

\textbf{User Story:} A user reads the file to the path \texttt{/:user/info.txt}.
Inside the file they see information regarding the user named \texttt{:user}. It
contains the following attributes:

\begin{itemize}
\item Full name
\item Screen name
\item Followers count
\item Statuses count
\item Location
\item Join date
\end{itemize}

\textbf{How to test:} The following should ouput "Lou Scoras".

\begin{alltt}
    \$ cat /ljsc/info.txt | grep -i 'Full-Name' | cut -d: -f2 
\end{alltt}

%%%%%%%%%%%%%%%%%%%%%%%%%%%%%%%%%%%%%%%%%%%%%%%%%%%%%%%%%%%%%%%%%%%%%%%%%%%%%%%%
\item Get a tweet given a particular id\label{req:get-a-tweet}

\textbf{User Story:} A user reads the file \texttt{/tweets/:tweet\_id.txt}.
The contents of the file will have the body of the tweet, which will contain:

\begin{itemize}
\item Message
\item Date
\item Author
\end{itemize}

If the user wants only the message contents, they can read the contents of
\texttt{/tweets/:tweet\_id/body.txt} alone.

\textbf{How to test:} Using the following command on predetermined tweet
\textit{n} yields ``This is a canned tweet.``

\begin{alltt}
    \$ cat /tweets/n/body.txt
\end{alltt}

\subsubsection{Getting lists of tweets}
%%%%%%%%%%%%%%%%%%%%%%%%%%%%%%%%%%%%%%%%%%%%%%%%%%%%%%%%%%%%%%%%%%%%%%%%%%%%%%%%
\item Get a user's latest tweets\label{req:get-latest}

\textbf{User Story:} A user lists the contents of the directory
\texttt{/:user/tweets/latest}. For each tweet in the timeline there will be a
file in the directory, \texttt{:tweet\_id.txt}. The contents of this file are as
described in SR-\ref{req:get-a-tweet}.

\textbf{How to test:} Take any tweet in a user's timeline within the last $n$
tweets, all of which are considered the latest. Let this tweets id be
\textit{tweet\_id}. The following command should then output $1$.

\begin{alltt}
    \$ ls -1 /vegasfs/tweets/latest | grep -i "\textit{tweet\_id}.txt" | wc -l
\end{alltt}

%%%%%%%%%%%%%%%%%%%%%%%%%%%%%%%%%%%%%%%%%%%%%%%%%%%%%%%%%%%%%%%%%%%%%%%%%%%%%%%%
\item Get a user's @mentions

\textbf{User Story:} This directory returns a user's mentions, up to 800 tweets (A
limitation by the API). To access the latest mentions, the user can list the
contents of \texttt{/:user/mentions} for the latest mentions for that user.
Also, they can fetch \texttt{/:user/mentions/:j-:k/} to list all the mentions
between the $j$th and $k$th mention in the timeline. Each file in these
directories will be \texttt{:tweet\_id.txt} and have the information as in
SR-\ref{req:get-a-tweet}.

\textbf{How to test:} For each file in the directory of mentions, we can ensure
that they all begin with `@` and contain \texttt{@:username}. The following
three commands should output the same value:

\begin{alltt}
    \$ cat /\textit{:user}/mentions/*.txt | wc -l
    \$ cat /\textit{:user}/mentions/*.txt | grep '^message:@' | wc -l
    \$ cat /\textit{:user}/mentions/*.txt | grep '^message:@' | grep '@\textit{:username}' | wc -l
\end{alltt}

\subsubsection{Updating tweets}
%%%%%%%%%%%%%%%%%%%%%%%%%%%%%%%%%%%%%%%%%%%%%%%%%%%%%%%%%%%%%%%%%%%%%%%%%%%%%%%%
\item Create a new tweet\label{req:create-tweet}

\textbf{User Story:} This feature allows creation of a new tweet by writing to a
file at location \texttt{/tweets/new.txt} with the contents of your tweet. The
date and author information will be appended automatically by the twitter
system.

\textbf{How to test:} Create a new tweet.

\begin{alltt}
    \$ vi /tweets/new.txt  # edit and save
\end{alltt}

The following command should contain the message entered in the previous step.

\begin{alltt}
    \$ cat \texttt{/:user/tweets/latest/}\$(ls -lt1 \texttt{/:user/tweets/latest} | head -n 1)
\end{alltt}


\subsubsection{Searching}
%%%%%%%%%%%%%%%%%%%%%%%%%%%%%%%%%%%%%%%%%%%%%%%%%%%%%%%%%%%%%%%%%%%%%%%%%%%%%%%%
\item Search for a hashtag

\textbf{User Story:} A user searches for a keyword among all tweets via the
hashtag metadata convention. When a user creates a tweet, they can prefix
keywords with the `\#` character. This search will target these tweets. The user
lists the contents of the directory \texttt{/:hashtag/tweets/}. There is one
tweet per file as described in SR-\ref{req:get-latest}.

\textbf{How to test:} Tweet a new tweet, or reuse a previous tweet for this test
as per SR-\ref{req:create-tweet}. In this tweet, create a hashtag with a GUID.
Then search for that tag as in the following command; it should output $1$.

\begin{alltt}
    \$ ls -1 \texttt{/\textit{GUID}/tweets/} | wc -l
\end{alltt}

\subsubsection{Content Negotiation}
%%%%%%%%%%%%%%%%%%%%%%%%%%%%%%%%%%%%%%%%%%%%%%%%%%%%%%%%%%%%%%%%%%%%%%%%%%%%%%%%

\item View tweets with images as images\label{req:images}

\textbf{User Story:} Users should be able to get attached media content from
tweets by changing the file extension. The if the user inspects a file
\texttt{tweet\_id.txt}, they should be able to see the picture embedded in the
tweet by opening \texttt{tweet\_id.png} for instance.

File formats that would be supported would be .png and .jpg. All embedded urls
using the yfrog service will be detected.

\textbf{How to test:} There are command-line utilities to inspect meta data
about an image file. One of these could be used to ensure that the data returned
is a valid image format. Also, we can use the UNIX \texttt{file} command to do
something similar:

\begin{alltt}
    \$ file /:user/tweets/1234.png # should return png
\end{alltt}

%%%%%%%%%%%%%%%%%%%%%%%%%%%%%%%%%%%%%%%%%%%%%%%%%%%%%%%%%%%%%%%%%%%%%%%%%%%%%%%%
\item Open a link for a given tweet as the expanded web page

\textbf{User Story:} If the embedded url is not detected as an image as in
SR-\ref{req:images}, it will be considered a normal url. If the user opens up
\texttt{tweet\_id.html}, it will return the contents of fetching that url
directly.

\textbf{How to test:} Open the file \texttt{tweet\_id.html} as described in the
story. The browser should show you the indicated page.

\end{Requirements}

\subsection{Optional Requirements}

\setcounter{requirements}{1}
\renewenvironment{Requirements}
  {\begin{list}{OR-\arabic{requirements}.}%
               {\usecounter{requirements}}}%
  {\end{list}}

\begin{Requirements}

\subsubsection{Advanced Search}
%%%%%%%%%%%%%%%%%%%%%%%%%%%%%%%%%%%%%%%%%%%%%%%%%%%%%%%%%%%%%%%%%%%%%%%%%%%%%%%%

\item Search for tweets in a geographical area

\textbf{User Story:} A user searches for nearby tweets by giving geotag
information.

The directory \texttt{/search/geo/:lat\_:long} will contain a number of files
with tweet information as in SR-\ref{req:get-latest}.

\textbf{How to test:} To test, the user would list the contents of the file
\texttt{/search/geo/xxx\_yyy/} and inspect the first file. This requirement will
be manually verified by inspecting the metadata for that tweet.

\subsubsection{Caching}

%%%%%%%%%%%%%%%%%%%%%%%%%%%%%%%%%%%%%%%%%%%%%%%%%%%%%%%%%%%%%%%%%%%%%%%%%%%%%%%%
\item Requests for tweets are cached between calls

\textbf{User Story:} A user gets information as in any of the static
requirements as above. If possible, the system should not re-fetch the entire
contents of the data if the cached version is fresh.

\textbf{How to test:} To be determined.

\end{Requirements}
