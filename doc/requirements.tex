\section{Requirements Analysis}

\subsection{Static Requirements}

In the following sections we will outline the project requirements. Each
requirement will have a short name, which is indicative of purpose for the
feature and is intended to for identifying the requirement. In addition, there
is a short user story, which explains the functionality in further detail.
Finally, each requirement will have a section which describes the acceptance
test which must pass for the requirement to be accepted.

The section on testing bares further discussion of terminology and notation.
Because the goal of the project is to make a file system for a POSIX
environment, it seems only right to define the desired operation in terms of
standard *NIX command line tool. Although for the project we will be creating an
automated test suite for all specified tests, it should also be possible to
verify correct operation using only tools on the command line.

As such, we will make heavy use of the following tools:

\begin{figure}[H]
\centering
\begin{tabular}{|c|c|}
\hline
COMMAND & DESCRIPTION \\\hline

ls -1 & Output each file in the directory on per line.\\

cat \textit{filename} & Output the contents of \textit{filename} to standard
output.\\

cut -d \textit{sep} -f \textit{field} & Output only the field numbered
\textit{field} using \textit{sep} to separate columns.\\

grep -i \textit{regex} & Output lines from the input that match \textit{regex}.
Case insensitive.\\

wc -l & Read the input and output the number of lines.\\\hline
\end{tabular}
\end{figure}

We assume familiarity with regular expressions and with UNIX
pipes\footnote{TODO: Add references for refreshers}. In user stories, any path
component prefixed with a ':' is to be considered a wildcard, and will be
referenced by the name following the symbol.

\hrulefill

\newcounter{requirements}
\newenvironment{Requirements}
  {\begin{list}{SR-\arabic{requirements}.}%
               {\usecounter{requirements}}}%
  {\end{list}}

\begin{Requirements}

\item Get information for a user.

\textbf{User Story:} A user reads the file to the path \texttt{/:user/info.txt}.
Inside the file they see information regarding the user named \texttt{:user}. It
contains the following attributes:

\begin{itemize}
\item Full name
\item Screen name
\item Followers count
\item Statuses count
\item Location
\item Join date
\end{itemize}

\textbf{How to test:} The following should ouput "Lou Scoras".

\begin{alltt}
    \$ cat /ljsc/info.txt | grep -i 'Full-Name' | cut -d: -f2 
\end{alltt}

\item Get a tweet given a particular id\label{get-a-tweet}

\textbf{User Story:} A user reads the file \texttt{/tweets/:tweet\_id.txt}.
The contents of the file will have the body of the tweet, which will contain:

\begin{itemize}
\item Message
\item Date
\item Author
\end{itemize}

If the user wants only the message contents, they can read the contents of
\texttt{/tweets/:tweet\_id/body.txt} alone.

\textbf{How to test:} Using the following command on predetermined tweet
\textit{n} yields ``This is a canned tweet.``

\begin{alltt}
    \$ cat /tweets/n/body.txt
\end{alltt}

%%%%%%%%%%%%%%%%%%%%%%%%%%%%%%%%%%%%%%%%%%%%%%%%%%%%%%%%%%%%%%%%%%%%%%%%%%%%%%%%

% SR_3: Get a user's latest tweets
% User Story: A user reads the file user/tweets/latest.txt
% The contents of the text file contain the latest tweet, which will contain:
% -Message
% -Date/Time posted
% -Author
% 
% Additionally, just the message body can be accessed by
% user/tweets/latest/body.txt
 
% How to test: Getting the above requirement's tweet at
% user/tweets/1.txt and user/tweets/latest.txt should be equal. (You can
% use http://api.twitter.com/1/statuses/user_timeline.json?count=1 to
% populate latest tweet)

\item Get a user's latest tweets

\textbf{User Story:} A user lists the contents of the directory
\texttt{/user/tweets/latest}. For each tweet in the timeline there will be a
file in the directory, \texttt{:tweet\_id.txt}. The contents of this file are as
described in SR-\ref{get-a-tweet}.

\textbf{How to test:} Take any tweet in a user's timeline within the last $n$
tweets, all of which are considered the latest. Let this tweets id be
\textit{tweet\_id}. The following command should then output $1$.

\begin{alltt}
    \$ ls -1 /vegasfs/tweets/latest | grep -i "\textit{tweet\_id}.txt" | wc -l
\end{alltt}

% SR_4 Get a user's @ mentions
% User Story: This file returns a user's mentions, up to 800 tweets.(A
% limitation by the API)
% To access the number of mentions, the user can access
% user/tweets/:number_of_tweets/mentions.txt where number_of_tweets is
% the number of tweets to fetch, from 1-800. This will point you to a
% text file containing all the @mentions with:
% -Message
% -Date/Time posted
% -Author
% Additionally, going to user/tweets/:number_of_tweets/mentions/body.txt
% returns the number of latest mentions with just the body of the
% message.
% Going to user/tweets/:number_of_tweets/mentions/author.txt returns a
% list of authors of mention tweets :number_of_tweets long.
% 
% How to test:
% (I can't think of a good test for this one...perhaps grepping all
% user's tweets for @:username? Seems like a stupid test to me...)
% 
% SR_5 Create a new tweet
% User Story: This file creates a new tweet via POST action data to the
% URI. this is done by writing to a file at location user/tweets/new
% with the contents of your tweet. The date and author information will
% be appended automatically by the twitter system.
% 
% How to test:
% Creating a new tweet and accesing /:user/tweets/1.txt matches with the
% information posted to the service.

%%%%%%%%%%%%%%%%%%%%%%%%%%%%%%%%%%%%%%%%%%%%%%%%%%%%%%%%%%%%%%%%%%%%%%%%%%%%%%%%

\item Search for a hashtag
\item Search for tweets in a geographical area

\item View tweets with images as images
\item Open a link for a given tweet as the expanded web page

\end{Requirements}

\subsection{Optional Requirements}
