\section{Tests}

Tests for the VegasFS projects are provided as executable specifications written in
RSpec, a behavior-driven development tool for Ruby. These specs were used in
the development of the project. They can be used to verify the correct
implementation of the Vegas driver and several of the features exposed via the
Sinatra layer.

To run the specs, simply execute the default Rake\footnote{Ruby's equivalent to
make} task at the command line. You can do this by typing \texttt{rake} from the
command line at the project root. The source code for these specifications can
be found in the \texttt{spec} directory.

As an example of what the specs look like, we have included an excerpt the
specification for the driver below. Also, recall that in
section~\ref{sec:requirements} we have also included how to manually test the
system from the command line.

\begin{verbatim}
describe VegasFS::Driver do
  describe "connected to localhost port 777" do
    before do
      @vegas = VegasFS::Driver.connect(:host => 'localhost', :port => 777)
    end

    describe "accessing a file" do
      it "should submit a GET request on read_file calls" do
        stub_request(:get, 'localhost:777/foo/bar')
        @vegas.read_file("/foo/bar")

        a_request(:get, "http://localhost:777/foo/bar").should have_been_made
      end

      it "should return the body contents of a GET request" do
        stub_request(:get, 'localhost:777/foo/bar').to_return(:body => "Baz!")
        @vegas.read_file("/foo/bar").should == "Baz!"
      end
    end
  end
end
\end{verbatim}

In the example we first set up an instance of the Vegas router. Then we specify
that whenever fuse calls Vegas to read a file, that a GET request should be sent
out to the corresponding URI path on the Sinatra handler.

